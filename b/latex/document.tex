\documentclass{article}

% Packages
\usepackage[utf8]{inputenc}
\usepackage{amsmath}
\usepackage{graphicx}

% Metadata
\title{LaTeX Exercise}
\author{Jaan Tollander de Balsch}
\date{\today}

% Referencing
%\usepackage[backend=biber,style=numeric]{biblatex}
%\addbibresource{bibliography.bib}

\usepackage[backend=bibtex,style=verbose-trad2]{biblatex}
\bibliography{bibliography}


%% Document
\begin{document}

% First page
\clearpage
\maketitle
\thispagestyle{empty}
\newpage


% Second page
\thispagestyle{empty}
\tableofcontents{}
\newpage


% Third page
\section{Maths}
\setcounter{page}{1}

\subsection{Pythagorean Theorem}
Pythagorean theorem $a^2+b^2=c^2$ states that given a right triangle with two sides $\mathbf{a}$ and $\mathbf{b}$ and hypotenuse $\mathbf{c}$, the combined area of squares $a^2$ and $b^2$ equals to the area of the square $c^2$.

Centered equation

\begin{align}
f(x)=x^2
\end{align}


\section{Figure}
\begin{figure}[h]
\includegraphics[width=\textwidth]{800x400.png}
    \caption{Here is my caption. \cite{htmlcom}}
\end{figure}


\section{Table}
\begin{table}[h]
\centering
\begin{tabular}{|c c c c|} 
 \hline
 1 & 6 & 87837 & 787 \\ 
 2 & 7 & 78 & 5415 \\
 3 & 545 & 778 & 7507 \\
 4 & 545 & 18744 & 7560 \\
 \hline
\end{tabular}
\caption{My table}
\label{table:1}
\end{table}


\section{Referencing}
Reference to the figure \cite{htmlcom}

\printbibliography

\end{document}
